\documentclass[]{UCD_CS_47360_Report}
\usepackage{graphicx}
\usepackage{hyperref}


%%%%%%%%%%%%%%%%%%%%%%
%%% Input project details

\def\studentname{John Jones} % Edit with your names
\def\projecttitle{Predicting Dublin Bus Journey Times} % Edit with your project title
\def\groupnum{3} % Edit with your group number

\begin{article}
\begin{document}

\maketitle

%%%%%%%%%%%%%%%%%%%%%%
%%% Your Project Specification

\chapter*{Project Specification}

Your project specification goes here.


%%%%%%%%%%%%%%%%%%%%%%
%%% Your Abstract here

\begin{abstract}

What is an abstract? The abstract should provide a short overview of your project that enables a reader to decide if your report is of interest to them or not. It should be concise, to-the-point and interesting. Avoid making it read like a verbose table of contents! Avoid references, jargon or acronyms, as the reader may not be familiar with them. An abstract usually contains a brief description of:

\begin{itemize}
\item The project and its context;
\item How the project work was carried out;
\item The major findings or results.
\end{itemize}

One paragraph is plenty! The main thing to remember is the principle that the abstract must be short, and a person reading it should be able to determine if they want to read more. For example, if your project involves building a compiler for Java, and a major section of your work is focussed on developing an efficient parser (rather than say code-generation), make this clear in the abstract. Then a reader who is interested in efficient parsing techniques knows that your report may be of interest to them.

\end{abstract}
\newpage


%%%%%%%%%%%%%%%%%%%%%%
%%% Acknowledgments

\chapter*{Acknowledgments}

In your Acknowledgments section, give credit to all the people who helped you in your project.


%%%%%%%%%%%%%%%%%%%%%%
%%% Table of Content

\tableofcontents\pdfbookmark[0]{Table of Contents}{toc}\newpage
\newpage


%%%%%%%%%%%%%%%%%%%%%%
%%% Introduction

\chapter{Introduction}



\chapter{\label{chapter2} Description of Final Product}

\section{This is a section}
\subsection{This is a subsection}

\chapter{\label{chapter3} Development Approach}
Here is an example of a numbered list.
\begin{enumerate}
\item {\sl item 1:} This is a numbered list
\item {\sl  item 2:} This is a numbered list
\item {\sl item 3} This is a numbered list
\end{enumerate}

\chapter{\label{chapter4}Technical Approach}
Here is an example of using bullet points.
\begin{itemize}
\item item 1
\item item 2
\item item 3
\end{itemize}


\chapter{\label{chapter5} Testing and Evaluation}
Here is an example of inserting a figure in various sizes.
\begin{figure}[h]
\centering
\fboxsep 2mm
\framebox{
	\includegraphics[width=6cm]{UCD_Logo} 
	\includegraphics[width=3cm]{UCD_Logo} 
	\includegraphics[width=1.5cm]{UCD_Logo} 
	\includegraphics[width=0.75cm]{UCD_Logo} 
	\includegraphics[width=0.375cm]{UCD_Logo}
}
\caption{\label{fig:logo} Logo of the UCD Department of Computer Science displayed at various size.}
\end{figure}

\chapter{\label{chapter6} Major Contributions}
If you wish to print a short excerpt of your source code,  ensure that you are using a fixed-width sans-serif font such as the Courier font. Your code will be properly indented and will appear as follows:

\begin{verbatim}
let names = ["Science", "Computer", "of", "School", "UCD"]
func backwards(s1: String, s2: String) -> Bool
{
   return s1 > s2
}
var reversed = sorted(names, backwards)
// reversed is equal to ["UCD", "School", "of", "Computer", "Science"]
\end{verbatim}

\chapter{\label{chapter7} Background Research}
This is an example of citing a paper which is in the bibliography. Check tutorials for other approaches to this ~\cite{DAWSON:2000}.



\chapter{\label{chapter8} Critical Evaluation & Future Work}
\section{This is a section}
\subsection{This is a subsection}




%%%% ADD YOUR BIBLIOGRAPHY HERE
\newpage
\begin{thebibliography}{99}
\bibitem{DAWSON:2000} Christian Dawson. \emph{The Essence of Computing Projects -- A Student's Guide}. 192 pages. ISBN: 013021972X. Pearson Education, 2000.
\end{thebibliography}
\label{endpage}



\end{document}

\end{article}
